\chapter{Half Wave Rectifier}
\vspace{-1cm}

%--------------------------AIM-----------------------------
\section{Aim}
\hspace{1cm}Single Phase Half Wave Uncontrolled and Controlled Rectifier

%--------------------------SOFTWARE USED-----------------------------
\section{Software Used}
\hspace{1cm}MATLAB R2020a
%-------------------------THEORY---------------------------
\section{Theory}
\hspace{\parindent}
A rectifier is a device that converts alternating current (AC) to direct
current (DC). It is done by using a diode or a group of diodes.
A half wave rectifier is defined as a type of rectifier that only allows
one half-cycle of an AC voltage waveform to pass, blocking the other
half-cycle. Half-wave rectifiers are used to convert AC voltage to DC
voltage, and only require a single diode to construct.

\hspace{\parindent}
\textbf{Single Phase Half Wave Uncontrolled Rectifier:}
This rectifier comprises of an AC source, a diode and a load. The
diode gets forward biased during the positive half cycle of the AC
source, and the circuit conducts. During the negative half cycle, the
diode becomes reverse biased and blocks current.

\hspace{\parindent}
\textbf{Single Phase Half Wave Controlled Rectifier:}
This rectifier comprises of an AC source, a Thyristor/SCR and a
load. The key difference here, is the presence of the thyristor/SCR,
which conducts only when gate pulses at a firing angle $ \alpha $ are applied
to it. The SCR automatically turns off when its voltage is reverse
biased for a period longer than the SCR turn off time and its current
falls below holding current.


% \vspace{1cm}

%----------------------Theoretical Calculations----------------------
\section{Theoretical Calculations}
\hspace{\parindent}
For an R load, average output voltage and current for a controlled
half wave rectifier are given by:

$$
    V_{o,avg} =
    V_{phase}
    \sqrt{2(1 + cos\alpha)2\pi} =
    V_m(1 + cos\alpha)
    2\pi
$$
$$
    I_{o,avg} =
    V_oR
$$

Where $ \alpha $ is the firing angle of the thyristor. For uncontrolled recti-
fiers, the thyristor is replaced by a diode, and $ \alpha $ = 0.

For a single phase half wave uncontrolled rectifier

\pagebreak

\section{Single Phase Half Wave Uncontrolled Rectifier with R load}

\subsection{Circuit used for simulation}

% figure that is centered on the page
\begin{figure}[h]
    \centering
    \includegraphics[width=1.0\textwidth]{images/experiment-1/circuit-diagram-experiment-01.png}
    \caption{Circuit for Single Phase Half Wave Uncontrolled Rectifier with R load}
    \label{Fig_simulation_circuit_single-phase-half-wave-uncontrolled-rectifier-with-R-load}
\end{figure}

\subsection{Components Required}

\begin{table}[h]
    \renewcommand{\arraystretch}{1.3}
    \label{table_components_required_single-phase-half-wave-uncontrolled-rectifier-with-R-load}
    \centering
    \begin{tabular}{|c|c|c|c|}
        \hline
        Sr. No & Parameters                     & Ratings            & Quantity \\
        \hline
        \hline
        1      & AC Single Phase Voltage Source & 230V ($ V_{rms} $) & 1        \\
        \hline
        2      & Resistor                       & 10$ \Omega $       & 1        \\
        \hline
        3      & Diode                          & -                  & 1        \\
        \hline
        4      & Voltmeter                      & -                  & 2        \\
        \hline
        5      & Ammeter                        & -                  & 1        \\
        \hline
    \end{tabular}
    \caption{Components for Single Phase Half Wave Uncontrolled Rectifier with R load}

\end{table}




\subsection{Observations}

\begin{table}[h]
    \renewcommand{\arraystretch}{1.3}
    \label{table_observation_single-phase-half-wave-uncontrolled-rectifier-with-R-load}
    \centering
    \begin{tabular}{|c|c|c|}
        \hline
        Parameters                              & Theoretical Values & Simulation Values \\
        \hline
        \hline
        AC Input Voltage ($ V_{in,rms} $)       & 230V               & 230V              \\
        \hline
        Output Average Voltage ($ V_{o,avg} $)  & 103.53V            & 103.1V            \\
        \hline
        Output Average Current ($ I_{o,avg}  $) & 10.35A             & 10.31A            \\
        \hline
    \end{tabular}
    \caption{Observations for Single Phase Half Wave Uncontrolled Rectifier with R load}

\end{table}


Upon observation, it is discerned that the simulated values coincide precisely with the corresponding theoretical values. Owing to the resistive nature of the load, the output current is in phase with the output voltage. Analysis of the output voltage and current waveforms reveals that the diode conducts during the positive half-cycle of the AC source, while it becomes reverse-biased during the negative half-cycle.
Additionally, the DC output power was estimated to be 1063W, while the AC input power was found to be 2636W, resulting in an overall efficiency of 40.34\%.

\pagebreak

\subsection{Resultant Waveforms}

% figure that is centered on the page
\begin{figure}[h]
    \centering
    \includegraphics[width=1\textwidth]{images/experiment-1/circuit-scope-experiment-01.png}
    \caption{Scope Waveforms for Single Phase Half Wave Uncontrolled Rectifier with R load waveforms}
    \label{Fig_waveform_single-phase-half-wave-uncontrolled-rectifier-with-R-load}
\end{figure}

\pagebreak
%-----------------------circuit 2--------------------------
\section{Single Phase Half Wave Uncontrolled Rectifier with RL load}

\subsection{Circuit used for simulation}

% figure that is centered on the page
\begin{figure}[h]
    \centering
    \includegraphics[width=0.7\textwidth]{images/experiment-1/circuit-diagram-simulation-02.png}
    \caption{Circuit used for simulation}
    \label{Fig_simulation_circuit_single-phase-half-wave-uncontrolled-rectifier-with-RL-load}
\end{figure}

\subsection{Components Required}

\begin{table}[h]
    \renewcommand{\arraystretch}{1.3}
    \label{table_components_required_circuit_2}
    \centering
    \begin{tabular}{|c|c|c|c|}
        \hline
        Sr. No & Parameters                     & Ratings            & Quantity \\
        \hline
        \hline
        1      & AC Single Phase Voltage Source & 230V ($ V_{rms} $) & 1        \\
        \hline
        2      & Resistor                       & 10$ \Omega $       & 1        \\
        \hline
        3      & Inductor                       & 10mH               & 1        \\
        \hline
        4      & Diode                          & -                  & 1        \\
        \hline
        5      & Voltmeter                      & -                  & 2        \\
        \hline
        6      & Ammeter                        & -                  & 1        \\
        \hline
    \end{tabular}
    \caption{Components for Single Phase Half Wave Uncontrolled Rectifier with RL load}
\end{table}


\subsection{Observations}

\begin{table}[h]
    \renewcommand{\arraystretch}{1.3}
    \label{table_observation_2}
    \centering
    \begin{tabular}{|c|c|c|}
        \hline
        Parameters                              & Theoretical Values & Simulation Values \\
        \hline
        \hline
        AC Input Voltage ($ V_{in,rms} $)       & 230V               & 230V              \\
        \hline
        Output Average Voltage ($ V_{o,avg} $)  & 103.53V            & 100.8V            \\
        \hline
        Output Average Current ($ I_{o,avg}  $) & 10.35A             & 10.08A            \\
        \hline
        AC Input Power ($ P_{AC} $)             & 2389.5 (W)         & 2318 (W)          \\
        \hline
        DC Input Power ($ P_{DC} $)             & 1071.53 (W)        & 1017 (W)          \\
        \hline
        Efficiency (\%)                         & 44.84              & 43.84             \\
        \hline
    \end{tabular}
    \caption{Observations for Single Phase Half Wave Uncontrolled Rectifier with RL load}

\end{table}


The circuit's simulated values are in good agreement with the theoretical values. However, because the load contains an inductive component, the output current lags behind the output voltage. This lag causes the diode to conduct until the output current reaches zero, resulting in the output voltage becoming negative during this time period. The diode ceases to conduct once the output current reaches zero, causing the output voltage to return to zero.
The efficiency of uncontrolled rectifier with RL load is 44.84\%.
\pagebreak


\subsection{Resultant Waveforms}

% figure that is centered on the page
\begin{figure}[h]
    \centering
    \includegraphics[width=1\textwidth]{images/experiment-1/circuit-scope-simulation-02.png}
    \caption{Scope Waveforms for Single Phase Half Wave Uncontrolled Rectifier with RL load}
    \label{Fig_waveform_single-phase-half-wave-uncontrolled-rectifier-with-RL-load}
\end{figure}

\pagebreak

\section{Single Phase Half Wave Uncontrolled Rectifier with RL load and Freewheeling Diode}

\subsection{Circuit used for simulation}

% figure that is centered on the page
\begin{figure}[h]
    \centering
    \includegraphics[width=0.7\textwidth]{images/experiment-1/circuit-diagram-simulation-03.png}
    \caption{Circuit used for simulation}
    \label{Fig_simulation_circuit_single-phase-half-wave-uncontrolled-rectifier-with-RL-load-and-freewheeling-diode}
\end{figure}

\subsection{Components Required}

\begin{table}[h]
    \renewcommand{\arraystretch}{1.3}
    \label{table_components_required_circuit_3}
    \centering
    \begin{tabular}{|c|c|c|c|}
        \hline
        Sr. No & Parameters                     & Ratings            & Quantity \\
        \hline
        \hline
        1      & AC Single Phase Voltage Source & 230V ($ V_{rms} $) & 1        \\
        \hline
        2      & Resistor                       & 10$ \Omega $       & 1        \\
        \hline
        3      & Inductor                       & 10mH               & 1        \\
        \hline
        4      & Diode                          & -                  & 1        \\
        \hline
        5      & Voltmeter                      & -                  & 2        \\
        \hline
        6      & Ammeter                        & -                  & 1        \\
        \hline
    \end{tabular}
    \caption{Components for Single Phase Half Wave Uncontrolled Rectifier with RL load and Freewheeling Diode}
\end{table}


\subsection{Observations}

\begin{table}[h]
    \renewcommand{\arraystretch}{1.3}
    \label{table_observation_3}
    \centering
    \begin{tabular}{|c|c|c|}
        \hline
        Parameters                              & Theoretical Values & Simulation Values \\
        \hline
        \hline
        AC Input Voltage ($ V_{in,rms} $)       & 230V               & 230V              \\
        \hline
        Output Average Voltage ($ V_{o,avg} $)  & 103.53V            & 103V              \\
        \hline
        Output Average Current ($ I_{o,avg}  $) & 10.35A             & 10.3A             \\
        \hline
        AC Input Power ($ P_{AC}  $)            & 2389.5 (W)         & 2404 (W)          \\
        \hline
        DC Input Power ($ P_{DC}  $)            & 1071.53 (W)        & 1061 (W)          \\
        \hline
        Efficiency (\%)                         & 44.84              & 44.13              \\
        \hline
    \end{tabular}
    \caption{Observations for Single Phase Half Wave Uncontrolled Rectifier with RL load and Freewheeling Diode}

\end{table}



The simulated and calculated values demonstrate that the simulated output voltage is close to the calculated voltage, but the simulated output current varies from the calculated current. Additionally, the inclusion of the freewheeling diode causes the output current in the rectifier circuit to cut off abruptly when the source AC supply reaches zero volts, as the lagging current begins to flow through the freewheeling diode rather than the rectifier circuit.
The efficiency of uncontrolled rectifier with RL load with freewheeling diode is 44.13\%.

\pagebreak

\subsection{Resultant Waveforms}


% figure that is centered on the page
\begin{figure}[h]
    \centering
    \includegraphics[width=1\textwidth]{images/experiment-1/circuit-scope-simulation-03.png}
    \caption{Scope Waveforms for Single Phase Half Wave Uncontrolled Rectifier with RL load and Freewheeling Diode}
    \label{Fig_waveform_single-phase-half-wave-uncontrolled-rectifier-with-RL-load-and-freewheeling-diode}
\end{figure}


\pagebreak


%-----------------------circuit 2--------------------------
\section{Single Phase Half Wave Uncontrolled Rectifier with RLE load}

\subsection{Circuit used for simulation}

% figure that is centered on the page
\begin{figure}[h]
    \centering
    \includegraphics[width=0.7\textwidth]{images/experiment-1/circuit-diagram-simulation-04.png}
    \caption{Circuit used for simulation}
    \label{Fig_simulation_circuit_single-phase-half-wave-uncontrolled-rectifier-with-RLE-load}
\end{figure}

\subsection{Components Required}

\begin{table}[h]
    \renewcommand{\arraystretch}{1.3}
    \caption{Components for Single Phase Half Wave Uncontrolled Rectifier with RLE load
        load}
    \label{table_components_required_circuit_4}
    \centering
    \begin{tabular}{|c|c|c|c|}
        \hline
        Sr. No & Parameters                     & Ratings            & Quantity \\
        \hline
        \hline
        1      & AC Single Phase Voltage Source & 230V ($ V_{rms} $) & 1        \\
        \hline
        2      & Resistor                       & 10$ \Omega $       & 1        \\
        \hline
        3      & Inductor                       & 10mH               & 1        \\
        \hline
        4      & Diode                          & -                  & 1        \\
        \hline
        5      & DC Source                      & 100V               & 1        \\
        \hline
        6      & Voltmeter                      & -                  & 2        \\
        \hline
        7      & Ammeter                        & -                  & 1        \\
        \hline
    \end{tabular}
\end{table}




\subsection{Observations}

\begin{table}[h]
    \renewcommand{\arraystretch}{1.3}
    \caption{Observations for Single Phase Half Wave Uncontrolled Rectifier with R
        load}
    \label{table_observation_4}
    \centering
    \begin{tabular}{|c|c|c|}
        \hline
        Parameters                              & Theoretical Values & Simulation Values \\
        \hline
        \hline
        AC Input Voltage ($ V_{in,rms} $)       & 230V               & 230V              \\
        \hline
        Output Average Voltage ($ V_{o,avg} $)  & 103.53V            & 101.1V            \\
        \hline
        Output Average Current ($ I_{o,avg}  $) & 10.35A             & 10.11A            \\
        \hline
    \end{tabular}
\end{table}


It is observed that the simulated values accurately match the the-
oretical values. As the load is resistive in nature, the output current
is in phase with output voltage. From the output voltage and cur-
rent waveforms, it can be deduced that the diode gets forward biased
during the positive half cycle of the AC source.

\pagebreak

\subsection{Resultant Waveforms}

% figure that is centered on the page
\begin{figure}[h]
    \centering
    \includegraphics[width=1\textwidth]{images/experiment-1/circuit-scope-simulation-04.png}
    \caption{Scope Waveforms for Single Phase Half Wave Uncontrolled Rectifier with RLE load waveforms}
    \label{Fig_waveform_single-phase-half-wave-uncontrolled-rectifier-with-RLE-load}
\end{figure}

\pagebreak


%-----------------------circuit 2--------------------------
\section{Single Phase Half Wave Controlled Rectifier with R load}

\subsection{Circuit used for simulation}

% figure that is centered on the page
\begin{figure}[h]
    \centering
    \includegraphics[width=0.7\textwidth]{images/experiment-1/circuit-diagram-simulation-05.png}
    \caption{Circuit used for simulation (Firing Angle = 30$ ^\circ $)}
    \label{Fig_simulation_circuit_single-phase-half-wave-controlled-rectifier-with-R-load}
\end{figure}

\subsection{Components Required}

\begin{table}[h]
    \renewcommand{\arraystretch}{1.3}
    \label{table_components_required_circuit_5}
    \centering
    \begin{tabular}{|c|c|c|c|}
        \hline
        Sr. No & Parameters                     & Ratings            & Quantity \\
        \hline
        \hline
        1      & AC Single Phase Voltage Source & 230V ($ V_{rms} $) & 1        \\
        \hline
        2      & Resistor                       & 10$ \Omega $       & 1        \\
        \hline
        3      & Inductor                       & 10mH               & 1        \\
        \hline
        4      & Diode                          & -                  & 1        \\
        \hline
        5      & Voltmeter                      & -                  & 2        \\
        \hline
        6      & Ammeter                        & -                  & 1        \\
        \hline
        7      & Thyristor                      & -                  & 1        \\
        \hline
    \end{tabular}
    \caption{Components for Single Phase Half Wave Controlled Rectifier with R load}

\end{table}



\subsection{Observations}

\begin{table}[h]
    \renewcommand{\arraystretch}{1.3}
    \label{table_observation_5}
    \centering
    \begin{tabular}{|c|c|c|}
        \hline
        Parameters                              & Theoretical Values & Simulation Values \\
        \hline
        \hline
        AC Input Voltage ($ V_{in,rms} $)       & 230V               & 230V              \\
        \hline
        Output Average Voltage ($ V_{o,avg} $)  & 96.6V              & 96.28V            \\
        \hline
        Output Average Current ($ I_{o,avg}  $) & 9.66A              & 9.628A            \\
        \hline
        AC Input Power ($ P_{AC}  $)            & 2214.44 (W)        & 2214 (W)          \\
        \hline
        DC Input Power ($ P_{DC}  $)            & 926.98 (W)         & 926.9 (W)         \\
        \hline
        Efficiency (\%)                         & 41.86              & 41.86             \\
        \hline
    \end{tabular}
    \caption{Observations for Single Phase Half Wave Controlled Rectifier with R load}

\end{table}


he simulation results indicate that the rectifier circuit is working as expected, and the output voltage and current follow the theoretical values accurately. As the load is purely resistive, the output current is in phase with the output voltage. The rectifier circuit is uncontrolled, and the output voltage waveform contains ripples that can be reduced by using a filter circuit.
The efficiency of controlled rectifier with R load is 41.86%.

\pagebreak


\subsection{Resultant Waveforms}

% figure that is centered on the page
\begin{figure}[h]
    \centering
    \includegraphics[width=1\textwidth]{images/experiment-1/circuit-scope-simulation-05.png}
    \caption{Scope Waveforms for Single Phase Half Wave Controlled Rectifier with R load}
    \label{Fig_waveform_single-phase-half-wave-controlled-rectifier-with-R-load}
\end{figure}

\pagebreak

\section{Single Phase Half Wave Controlled Rectifier with RL load}

\subsection{Circuit used for simulation}

% figure that is centered on the page
\begin{figure}[h]
    \centering
    \includegraphics[width=1.0\textwidth]{images/experiment-1/circuit-diagram-experiment-06.png}
    \caption{Circuit for Single Phase Half Wave Controlled Rectifier with RL load  (Firing Angle = 30$ ^\circ $)}
    \label{Fig_simulation_circuit_single-phase-half-wave-controlled-rectifier-with-RL-load}
\end{figure}

\subsection{Components Required}

\begin{table}[h]
    \renewcommand{\arraystretch}{1.3}
    \label{table_components_required_single-phase-half-wave-controlled-rectifier-with-RL-load}
    \centering
    \begin{tabular}{|c|c|c|c|}
        \hline
        Sr. No & Parameters                     & Ratings            & Quantity \\
        \hline
        \hline
        1      & AC Single Phase Voltage Source & 230V ($ V_{rms} $) & 1        \\
        \hline
        2      & Resistor                       & 10$ \Omega $       & 1        \\
        \hline
        3      & Inductor                       & 10mH               & 1        \\
        \hline
        4      & Diode                          & -                  & 1        \\
        \hline
        5      & Voltmeter                      & -                  & 2        \\
        \hline
        6      & Ammeter                        & -                  & 1        \\
        \hline
        7      & Thyristor                      & -                  & 1        \\
        \hline
    \end{tabular}
    \caption{Components for Single Phase Half Wave Controlled Rectifier with RL load}

\end{table}




\subsection{Observations}

\begin{table}[h]
    \renewcommand{\arraystretch}{1.3}
    \label{table_observation_single-phase-half-wave-controlled-rectifier-with-RL-load}
    \centering
    \begin{tabular}{|c|c|c|}
        \hline
        Parameters                              & Theoretical Values & Simulation Values \\
        \hline
        \hline
        AC Input Voltage ($ V_{in,rms} $)       & 230V               & 230V              \\
        \hline
        Output Average Voltage ($ V_{o,avg} $)  & 96.6V              & 93.36V            \\
        \hline
        Output Average Current ($ I_{o,avg}  $) & 9.66A              & 9.343A            \\
        \hline
    \end{tabular}
    \caption{Observations for Single Phase Half Wave Controlled Rectifier with RL load}

\end{table}


Upon providing the firing gate pulse to the thyristor, it is observed that the circuit begins conducting. Due to the presence of an inductive component in the load, the output current lags behind the output voltage, leading to the conduction of the diode until the output current approaches zero. Consequently, the output voltage becomes negative during this duration. After the output current falls to zero, the thyristor ceases conduction, and the output voltage returns to zero as well.
Additionally, the DC output power was estimated to be 872.3W, while the AC input power was found to be 2296W, resulting in an overall efficiency of 37.99\%.



% \pagebreak

\subsection{Resultant Waveforms}

% figure that is centered on the page
\begin{figure}[h]
    \centering
    \includegraphics[width=1\textwidth]{images/experiment-1/circuit-scope-experiment-06.png}
    \caption{Scope Waveforms for Single Phase Half Wave Controlled Rectifier with RL load}
    \label{Fig_waveform_single-phase-half-wave-controlled-rectifier-with-RL-load}
\end{figure}

\pagebreak

%-----------------------circuit 2--------------------------
\section{Single Phase Half Wave Controlled Rectifier with RLE load}

\subsection{Circuit used for simulation}

% figure that is centered on the page
\begin{figure}[h]
    \centering
    \includegraphics[width=0.7\textwidth]{images/experiment-1/circuit-diagram-simulation-07.png}
    \caption{Circuit used for simulation}
    \label{Fig_simulation_circuit_single-phase-half-wave-controlled-rectifier-with-RLE-load}
\end{figure}

\subsection{Components Required}

\begin{table}[h]
    \renewcommand{\arraystretch}{1.3}
    \label{table_components_required_circuit_7}
    \centering
    \begin{tabular}{|c|c|c|c|}
        \hline
        Sr. No & Parameters                     & Ratings            & Quantity \\
        \hline
        \hline
        1      & AC Single Phase Voltage Source & 230V ($ V_{rms} $) & 1        \\
        \hline
        2      & Resistor                       & 10$ \Omega $       & 1        \\
        \hline
        3      & Inductor                       & 10mH               & 1        \\
        \hline
        4      & Diode                          & -                  & 1        \\
        \hline
        5      & Voltmeter                      & -                  & 2        \\
        \hline
        6      & Ammeter                        & -                  & 1        \\
        \hline
        7      & Thyristor                      & -                  & 1        \\
        \hline
        8      & DC Source                      & 100V               & 1        \\
        \hline
    \end{tabular}
    \caption{Components for Single Phase Half Wave Controlled Rectifier with RLE load}

\end{table}

\pagebreak

\subsection{Observations}

\begin{table}[h]
    \renewcommand{\arraystretch}{1.3}
    \label{table_observation_7}
    \centering
    \begin{tabular}{|c|c|c|}
        \hline
        Parameters                              & Theoretical Values & Simulation Values \\
        \hline
        \hline
        AC Input Voltage ($ V_{in,rms} $)       & 230V               & 230V              \\
        \hline
        Output Average Voltage ($ V_{o,avg} $)  & 96.66V             & 155.1V            \\
        \hline
        Output Average Current ($ I_{o,avg}  $) & 9.66A              & 5.507A            \\
        \hline
        AC Input Power ($ P_{AC}  $)            & 2214.44 (W)        & 1266 (W)          \\
        \hline
        DC Input Power ($ P_{DC}  $)            & 926.98 (W)         & 853.9 (W)         \\
        \hline
        Efficiency (\%)                         & 41.86              & 67.43             \\
        \hline
    \end{tabular}
    \caption{Observations for Single Phase Half Wave Controlled Rectifier with RLE load}

\end{table}


Upon giving the firing gate pulse to the thyristor, the circuit is observed to initiate conduction. Once the circuit initiates conduction, its characteristics resemble those of an uncontrolled half wave rectifier with RLE load.
The efficiency of controlled rectifier
with RL load is 67.43\%.



% \pagebreak


\subsection{Resultant Waveforms}

% figure that is centered on the page
\begin{figure}[h]
    \centering
    \includegraphics[width=1\textwidth]{images/experiment-1/circuit-scope-simulation-07.png}
    \caption{Scope Waveforms for Single Phase Half Wave Controlled Rectifier with RLE load}
    \label{Fig_waveform_single-phase-half-wave-controlled-rectifier-with-RLE-load}
\end{figure}


\pagebreak





%-------------------------RESULTS---------------------------
\section{Results}
%------------SUB----------RESULTS---------------------------
\subsection{Theoretical Calculation}


\hspace{1.5cm} The resistance \textbf{R} in the circuit shown in Fig. 2.2 is considered as 20 $\Omega$, and the voltage (which is effectively the voltage across \textbf{R}) is considered as 50 V, 100 V, 150 V and 200 V in four steps as mentioned in the second column of Table-\ref{Table_simulation-result}. As per \textbf{Ohm's Law}, the current corresponding to all the four voltages are,
$$I=\frac{V}{R}=\frac{50}{20}= 2.5~A~~ (for~ V =50 V);~~~   = 5~A~~ (for~ V =100 V);~~~
    = 7.5~A~~ (for~ V =150 V);~~~= 10~A~~ (for~ V =200 V) $$


%------------SUB----------RESULTS---------------------------
\subsection{Simulation Results}
\hspace{1.5cm} The simulink file is run for 10 sec considering V=50 V, and and corresponding current seen in the display is noted in the fourth column of second row of Table-\ref{Table_simulation-result}. Similarly, all other three rows are filled.  Further, constantly varying ramp voltage is applied and the corresponding \textit{v-i} graph is plotted in Fig. \ref{Fig_v-i graph}.




\begin{table}[h]
    \renewcommand{\arraystretch}{1.3}
    \caption{Observations for Single Phase Half Wave Controlled Rectifier with RLE load}
    \centering
    \begin{tabular}{|c|c|c|c|}
        \hline
        \multirow{2}{*}{Sl No} & \multirow{2}{*}{Applied Voltage (V) in Volts} & \multicolumn{2}{c|}{Current (I) through R in Amps}             \\
        \cline{3-4}
                               &                                               & Theoretical                                        & Simulated \\
        \hline
        1                      & 50                                            & 5                                                  & 2.5       \\
        \hline
        2                      & 100                                           & 10                                                 & 5         \\
        \hline
        3                      & 150                                           & 15                                                 & 7.5       \\
        \hline
        4                      & 200                                           & 20                                                 & 10        \\
        \hline
    \end{tabular}
\end{table}


\section{Conclusion}
\hspace{\parindent}
The design of single phase half wave rectifiers, both controlled and
uncontrolled, with R, RL, RL with freewheeling diode, and RLE
loads were implemented successfully in MATLAB using Simulink.
The output waveforms for voltage and current were obtained in each
case, and a comparison between theoretically calculated and simu-
lated output parameter values was also performed.

\pagebreak