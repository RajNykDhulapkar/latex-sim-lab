\chapter{Full Wave Rectifier}
\vspace{-1cm}

%--------------------------AIM-----------------------------
\section{Aim}
\hspace{1cm}To design and implement Single Phase Full Wave Uncontrolled and Controlled rectifiers, and
simulate them with R and RL loads.

%--------------------------SOFTWARE USED-----------------------------
\section{Software Used}
\hspace{1cm}MATLAB R2020a
%-------------------------THEORY---------------------------
\section{Theory}
\hspace{\parindent}

Rectification is the process of converting an alternating current (AC) voltage source into a direct current (DC) voltage supply by using electronic components such as diodes, thyristors, and transistors. This process can take on many forms, including half-wave, full-wave, uncontrolled, and fully-controlled rectifiers. Half-wave rectifiers use only one diode to rectify the AC signal, resulting in a pulsating DC output. Full-wave rectifiers, on the other hand, use two diodes to rectify the AC signal, producing a smoother DC output. Uncontrolled rectifiers are simple rectifiers that use diodes to rectify the AC signal without any regulation, while fully-controlled rectifiers use thyristors to control the DC output voltage.

% \vspace{1cm}

%----------------------Theoretical Calculations----------------------
\section{Theoretical Calculations}
\hspace{\parindent}
In the theory of full-wave rectifiers, the average output voltage and current for controlled full-wave rectifiers with resistive (R) load can be calculated using the following equations:

\begin{equation}
    V_0 = \frac{V_{phase}}{\sqrt{2}}\frac{(1+\cos\alpha)}{\pi}
\end{equation}

\begin{equation}
    V_0 = \frac{V_m(1+\cos\alpha)}{2\pi}
\end{equation}

\begin{equation}
    I_0 = \frac{V_0}{R}
\end{equation}

where $\alpha$ is the thyristor's firing angle. It should be noted that in uncontrolled rectifiers, the thyristor is switched out for a diode and $\alpha$ is equal to 0.

For a single-phase full-wave uncontrolled rectifier, the average output voltage and current can be calculated as follows:

\begin{equation}
    V_0 = 2\sqrt{2}V_{rms}
\end{equation}

\begin{equation}
    I_0 = 2I_{rms}
\end{equation}

where $V_{rms}$ is the root mean square value of the input voltage and $I_{rms}$ is the root mean square value of the input current.

For a single-phase full-wave controlled rectifier, the average output voltage and current can be calculated as follows:

\begin{equation}
    V_0 = \frac{2\sqrt{2}V_{rms}(1-\cos\alpha)}{\pi}
\end{equation}

\begin{equation}
    I_0 = \frac{2I_{rms}(1-\cos\alpha)}{\pi}
\end{equation}

Consider an AC source with an RMS voltage of 230V and a resistive load of 10$ \Omega $. The output voltage and current for a single phase full-wave uncontrolled rectifier are given by:

\begin{align}
    V_{0} & = 207.07\mathrm{V} \\
    I_{0} & = 20.70\mathrm{A}
\end{align}

% Similarly, for a single phase half-wave controlled rectifier with a firing angle of $ \alpha $  = 30, the output voltage and current are given by:

% \begin{align}
%     V_{0} & = 193.2\mathrm{V} \\
%     I_{0} & = 19.32\mathrm{A}
% \end{align}

% The DC power output for a single phase full-wave uncontrolled rectifier is found to be 4173.849W, while for a single phase full-wave controlled rectifier with a firing angle of $ \alpha $ = 30, the DC power output is 3638.554W.

\pagebreak

%-----------------------circuit 1--------------------------
\section{Single Phase Full Wave Uncontrolled Rectifier
  with R load}

\subsection{Circuit used for simulation}

% figure that is centered on the page
\begin{figure}[h]
    \centering
    \includegraphics[width=0.7\textwidth]{images/experiment-2/circuit-diagram-simulation-01.png}
    \caption{Circuit used for simulation}
    \label{Fig_simulation_circuit_single-phase-full-wave-uncontrolled-rectifier-with-R-load}
\end{figure}

\subsection{Components Required}

\begin{table}[h]
    \renewcommand{\arraystretch}{1.3}
    \label{table_components_required_circuit_1}
    \centering
    \begin{tabular}{|c|c|c|c|}
        \hline
        Sr. No & Parameters                     & Ratings            & Quantity \\
        \hline
        \hline
        1      & AC Single Phase Voltage Source & 230V ($ V_{rms} $) & 1        \\
        \hline
        2      & Resistor                       & 10$ \Omega $       & 1        \\
        \hline
        3      & Diode                          & -                  & 1        \\
        \hline
        4      & Voltmeter                      & -                  & 2        \\
        \hline
        5      & Ammeter                        & -                  & 1        \\
        \hline
    \end{tabular}
    \caption{Components for Single Phase Full Wave Uncontrolled Rectifier with R load}

\end{table}




\subsection{Observations}

\begin{table}[h]
    \renewcommand{\arraystretch}{1.3}
    \label{table_observation_circuit_1}
    \centering
    \begin{tabular}{|c|c|c|}
        \hline
        Parameters                              & Theoretical Values & Simulation Values \\
        \hline
        \hline
        AC Input Voltage ($ V_{in,rms} $)       & 230V               & 230V              \\
        \hline
        Output Average Voltage ($ V_{o,avg} $)  & 207.07V            & 205.4V            \\
        \hline
        Output Average Current ($ I_{o,avg}  $) & 20.70A             & 20.54V            \\
        \hline
        DC Input Power ($ P_{DC}  $)            & 4218.916W          & 4173W             \\
        \hline
        Efficiency (\%)                         & 79.73              & 79.73             \\
        \hline
    \end{tabular}
    \caption{Observations for Single Phase Full Wave Uncontrolled Rectifier with R load}

\end{table}


The results of the simulation indicate that the theoretical values and the simulated values are in good agreement. The load in the circuit is resistive, which means that the output current and voltage are in phase. The full-wave rectification of the AC signal produces an output DC signal with a frequency that is twice the frequency of the input signal. This increased frequency can be beneficial in applications that require higher-frequency DC power.
\pagebreak

\subsection{Resultant Waveforms}

% figure that is centered on the page
\begin{figure}[h]
    \centering
    \includegraphics[width=1\textwidth]{images/experiment-2/circuit-scope-simulation-01.png}
    \caption{Scope Waveforms for Single Phase Full Wave Uncontrolled Rectifier with R load waveforms}
    \label{Fig_waveform_single-phase-full-wave-uncontrolled-rectifier-with-R-load}
\end{figure}

\pagebreak

\section{Single Phase Full Wave Controlled Rectifier with
  R load}

\subsection{Circuit used for simulation}

% figure that is centered on the page
\begin{figure}[h]
    \centering
    \includegraphics[width=0.7\textwidth]{images/experiment-2/circuit-diagram-simulation-02.png}
    \caption{Circuit used for simulation}
    \label{Fig_simulation_circuit_single-phase-full-wave-controlled-rectifier-with-R-load}
\end{figure}

\subsection{Components Required}

\begin{table}[h]
    \renewcommand{\arraystretch}{1.3}
    \label{table_components_required_circuit_2}
    \centering
    \begin{tabular}{|c|c|c|c|}
        \hline
        Sr. No & Parameters                     & Ratings            & Quantity \\
        \hline
        \hline
        1      & AC Single Phase Voltage Source & 230V ($ V_{rms} $) & 1        \\
        \hline
        2      & Resistor                       & 10$ \Omega $       & 1        \\
        \hline
        3      & Inductor                       & 10mH               & 1        \\
        \hline
        4      & Diode                          & -                  & 1        \\
        \hline
        5      & Voltmeter                      & -                  & 2        \\
        \hline
        6      & Ammeter                        & -                  & 1        \\
        \hline
    \end{tabular}
    \caption{Components for Single Phase Half Wave Uncontrolled Rectifier with RL load}
\end{table}


\subsection{Observations}

\begin{table}[h]
    \renewcommand{\arraystretch}{1.3}
    \label{table_observation_2}
    \centering
    \begin{tabular}{|c|c|c|}
        \hline
        Parameters                              & Theoretical Values & Simulation Values \\
        \hline
        \hline
        AC Input Voltage ($ V_{in,rms} $)       & 230V               & 230V              \\
        \hline
        Output Average Voltage ($ V_{o,avg} $)  & 193.2V             & 190.9V            \\
        \hline
        Output Average Current ($ I_{o,avg}  $) & 19.32A             & 19.09A            \\
        \hline
        AC Input Power ($ P_{AC} $)             & 2389.5 (W)         & 2318 (W)          \\
        \hline
        DC Input Power ($ P_{DC} $)             & 1071.53 (W)        & 1017 (W)          \\
        \hline
        Efficiency (\%)                         & 44.84              & 43.84             \\
        \hline
    \end{tabular}
    \caption{Observations for Single Phase Half Wave Uncontrolled Rectifier with RL load}

\end{table}


The circuit is designed to operate as a full bridge uncontrolled rectifier, but it will not produce any output voltage until firing pulses are applied to the thyristor gates. Once the thyristors are triggered, the circuit simulates a full bridge uncontrolled rectifier. The simulated values match the theoretical values closely. Because the load in the circuit is resistive, the output current is in phase with the output voltage. The output DC signal has a frequency that is twice that of the input AC signal.
The efficiency of uncontrolled rectifier with RL load is 44.84\%.
\pagebreak


\subsection{Resultant Waveforms}

% figure that is centered on the page
\begin{figure}[h]
    \centering
    \includegraphics[width=1\textwidth]{images/experiment-2/circuit-scope-simulation-02.png}
    \caption{Scope Waveforms for Single Phase Half Wave Uncontrolled Rectifier with RL load}
    \label{Fig_waveform_single-phase-full-wave-controlled-rectifier-with-R-load}
\end{figure}

\pagebreak

\section{Single Phase Full Wave Controlled Rectifier with RL load}

\subsection{Circuit used for simulation}

% figure that is centered on the page
\begin{figure}[h]
    \centering
    \includegraphics[width=0.7\textwidth]{images/experiment-2/circuit-diagram-simulation-03.png}
    \caption{Circuit used for simulation}
    \label{Fig_simulation_circuit_single-phase-full-wave-controlled-rectifier-with-RL-load}
\end{figure}

\subsection{Components Required}

\begin{table}[h]
    \renewcommand{\arraystretch}{1.3}
    \label{table_components_required_circuit_3}
    \centering
    \begin{tabular}{|c|c|c|c|}
        \hline
        Sr. No & Parameters                     & Ratings            & Quantity \\
        \hline
        \hline
        1      & AC Single Phase Voltage Source & 230V ($ V_{rms} $) & 1        \\
        \hline
        2      & Resistor                       & 10$ \Omega $       & 1        \\
        \hline
        3      & Inductor                       & 10mH               & 1        \\
        \hline
        4      & Diode                          & -                  & 1        \\
        \hline
        5      & Voltmeter                      & -                  & 2        \\
        \hline
        6      & Ammeter                        & -                  & 1        \\
        \hline
    \end{tabular}
    \caption{Components for Single Phase Half Wave Uncontrolled Rectifier with RL load and Freewheeling Diode}
\end{table}


\subsection{Observations}

\begin{table}[h]
    \renewcommand{\arraystretch}{1.3}
    \label{table_observation_3}
    \centering
    \begin{tabular}{|c|c|c|}
        \hline
        Parameters                              & Theoretical Values & Simulation Values \\
        \hline
        \hline
        AC Input Voltage ($ V_{in,rms} $)       & 230V               & 230V              \\
        \hline
        Output Average Voltage ($ V_{o,avg} $)  & 103.53V            & 103V              \\
        \hline
        Output Average Current ($ I_{o,avg}  $) & 10.35A             & 10.3A             \\
        \hline
        AC Input Power ($ P_{AC}  $)            & 2389.5 (W)         & 2266 (W)          \\
        \hline
        DC Input Power ($ P_{DC}  $)            & 1071.53 (W)        & 1015 (W)          \\
        \hline
        Efficiency (\%)                         & 44.84              & 44.8              \\
        \hline
    \end{tabular}
    \caption{Observations for Single Phase Half Wave Uncontrolled Rectifier with RL load and Freewheeling Diode}

\end{table}



Upon analysis, it has been observed that the simulated output voltage closely approximates the calculated voltage, whereas the simulated output current significantly deviates from the calculated current. The incorporation of the freewheeling diode results in a sudden cessation of output current in the rectifier circuit when the AC supply source drops to zero volts, as the lagging current shifts to flow through the freewheeling diode rather than the rectifier circuit.
The efficiency of uncontrolled rectifier with RL load with freewheeling diode is 44.8\%.

\pagebreak

\subsection{Resultant Waveforms}


% figure that is centered on the page
\begin{figure}[h]
    \centering
    \includegraphics[width=1\textwidth]{images/experiment-2/circuit-scope-simulation-03.png}
    \caption{Scope Waveforms for Single Phase Half Wave Uncontrolled Rectifier with RL load and Freewheeling Diode}
    \label{Fig_waveform_single-phase-full-wave-controlled-rectifier-with-RL-load}
\end{figure}


\pagebreak



\section{Conclusion}


\hspace{\parindent}

Utilizing the MATLAB Simulink platform, the design of single phase full wave rectifiers with both controlled and uncontrolled R and RL loads was carried out with remarkable success. Voltage and current output waveforms were attained and output parameter values, both theoretically calculated and simulated, were juxtaposed.
The full-wave uncontrolled rectifier's efficiencies with R and RL load are measured to be 89.32\%. Furthermore, the full-wave controlled rectifier's efficiencies with R and RL load were measured to be 83.41\% and 81.33\%, respectively.

In conclusion, the full-wave rectifier is a more efficient and practical alternative to the half-wave rectifier because it produces a smoother DC output, has a higher output voltage and lower output ripple. The four diodes used in a bridge configuration ensure that the input AC voltage is rectified, producing a DC output voltage across the output terminals.
This process of rectification produces a unidirectional DC output signal that can be used for various applications.
The average output voltage and current for full-wave rectifiers with R load can be calculated using the above equations. It is important to note that these calculations are based on idealized conditions and practical circuits may have additional factors that affect their performance.
\pagebreak
